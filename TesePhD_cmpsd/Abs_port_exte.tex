\input{includes/preamble}
\input{includes/lstdefines}
%----------------------------------------------------------------------------------------
%	COVER PAGE
%----------------------------------------------------------------------------------------

% ====================================== Font Sizes
\def\FontXL{% 18 pt normal
  \usefont{T1}{cmr}{m}{n}\fontsize{19.28pt}{18pt}\selectfont}
% \usefont{T1}{pvh}{m}{n}\fontsize{16pt}{16pt}\selectfont}
\def\FontL{% 16 pt normal
  \usefont{T1}{cmr}{m}{n}\fontsize{17.28pt}{16pt}\selectfont}
% \usefont{T1}{pvh}{m}{n}\fontsize{16pt}{16pt}\selectfont}
\def\FontM{% 14 pt normal
  \usefont{T1}{cmr}{m}{n}\fontsize{14pt}{14pt}\selectfont}
% \usefont{T1}{phv}{m}{n}\fontsize{14pt}{14pt}\selectfont}
\def\FontS{% 12 pt normal
  \usefont{T1}{cmr}{m}{n}\fontsize{12pt}{12pt}\selectfont}
% \usefont{T1}{phv}{m}{n}\fontsize{12pt}{12pt}\selectfont}
\def\FontT{% 10 pt normal
  \fontsize{10pt}{10pt}\selectfont}

\newcommand*{\titleGP}{\begingroup % Create the command for including the title page in the document

% ====================================== Logo
%\noindent \includegraphics[width=5cm]{includes/LogoIST.pdf}
%\includegraphics[width=5cm]{includes/LogoIST.pdf}

\begin{tabular}{>{\raggedleft}m{5cm}>{\centering}m{\dimexpr\textwidth - 10cm\relax}>{\raggedright}m{5cm}}
    \includegraphics[width=\linewidth]{includes/LogoIST.pdf}%
    &
    &%
    \includegraphics[width=\linewidth]{includes/LogoPadova.jpg} %
 \end{tabular}

% ====================================== Cover information
\centering % Center all text

{\FontL \textbf{UNIVERSIDADE DE LISBOA}} \\
\vspace{10pt}
{\FontL \textbf{INSTITUTO SUPERIOR T\'{E}CNICO}} \\
\vspace{10pt}
{\FontM \textbf{Universit\`{a} degli Studi di Padova}} \\
\vspace{2cm}

{\FontXL \textbf{Tokamak Magnetic Control Simulation: Applications for JT-60SA and ISTTOK Operation.}} \\

\vspace{2cm}
{\FontM \textbf{Lilia Dom\'enica Corona Rivera}} \\
\vspace{2cm}
{\FontS %
\begin{tabular}{l}
\textbf{Supervisor:Prof. Hor\'acio Fernandes} \\
\textbf{Co-Supervisor: Prof. Nuno Cruz}\\
\textbf{External supervisor: Prof. Alfredo Pironti}\\
\end{tabular} } \\
\vspace{1.8cm}
{\FontM Thesis specifically prepared to obtain the PhD Degree in} \\
\vspace{1.8mm}
{\FontL \textbf{Technological Physics Engineering}} \\
\vspace{1.8cm}
{\FontM \textbf{Draft}} \\
%\vspace{1.8cm}
{\FontM \textbf{July 2020}} \\

\endgroup}
\newcommand{\todo}[1]{%
\textcolor{red}{@TODO: #1}
\GenericWarning{}{LaTeX Warning: You have things left to do!}
}%

\newcommand{\proton}[1]{%
    \shade[ball color=red] (#1) circle (.25);\draw (#1) node{$+$};
}

%\neutron{xposition,yposition}
\newcommand{\neutron}[1]{%
    \shade[ball color=green] (#1) circle (.25);
}

%\electron{xwidth,ywidth,rotation angle}
\newcommand{\electron}[3]{%
    \draw[rotate = #3](0,0) ellipse (#1 and #2)[color=blue];
    \shade[ball color=blue] (0,#2)[rotate=#3] circle (.1);
}

%\orbital{xwidth,ywidth,rotation angle}
\newcommand{\orbital}[3]{%
    \draw[rotate = #3](0,0) ellipse (#1 and #2)[color=blue];
}

\newcommand{\nucleus}[2]{%
    \neutron{#1+0.1,#2+0.3}
    \proton{#1+0,#2+0}
    \neutron{#1+0.3,#2+0.2}
    \proton{#1-0.2,#2+0.1}
    \neutron{#1-0.1,#2+0.3}
    \proton{#1+0.2,#2-0.15}
    \neutron{#1-0.05,#2-0.12}
    \proton{#1+0.17,#2+0.21}
}

\newcommand{\ion}[2]{%
    \neutron{#1+0.1,#2+0.3}
    \proton{#1+0,#2+0}
    \neutron{#1+0.3,#2+0.2}
    \proton{#1-0.2,#2+0.1}
}

%\curvearrow{position,radius, start angle, stop angle}
\newcommand{\curvearrow}[4]{%
  \coordinate (P) at ($(#1) + (#3:#2)$);
  \draw[thick, -latex] ($(#1) + (#3:#2)$(P) arc (#3:#4:#2);
}

\makenomenclature
%% This removes the main of the nomcl pack title:
\renewcommand{\nomname}{}
%% this modifies item separation:
\setlength{\nomitemsep}{8pt}
%----------------------------------------------
\usepackage{etoolbox}
\renewcommand{\nomgroup}[1]{%
\item[]\newpage\hspace*{-\leftmargin}%
\textbf{\Large
\ifstrequal{#1}{V}{List of Variables}{%
 \ifstrequal{#1}{A}{List of Abbreviations}{}}}%
}
%----------------------------------------------

\hyphenation{op-tical net-works semi-conduc-tor mi-nu-tos vo-lu-me la-bo-ra-to-ri-es a-na-ly-sis gas-e-ous}


\usepackage{enumitem}
\usepackage{notoccite}
\usepackage{longtable}
\usepackage{multirow}
\usepackage{lipsum}
\usepackage{xcolor,colortbl}
\definecolor{LightCyan}{rgb}{0.88,1,1}
\definecolor{amethyst}{rgb}{0.6, 0.4, 0.8}
\definecolor{color2}{RGB}{228, 206, 237 }
\definecolor{color1}{RGB}{202, 170, 229}
\definecolor{color3}{RGB}{237, 206, 233 }
\usepackage{mathtools}

\DeclareMathOperator*{\argmin}{arg\,min}
\DeclareMathOperator{\dist}{\mathit{dist}}

\begin{document}
	\tolerance=1
	\emergencystretch=\maxdimen
	\hyphenpenalty=10000
	\hbadness=10000
	
	{\fontsize{12}{12}\selectfont \textbf{ Simulação do Controlo Magnético do Tokamak: Aplicações para o JT-60SA e Operação do ISTTOK}}
	
\medskip

\setlength{\parskip}{1em}

	
	\textbf{Resumo alargado em Português}
	\smallskip
	
O controlo magnético para  plasmas de fusão nuclear é um dos principais problemas a ser resolvidos em dispositivos de confinamento magnético como os tokamaks. Um dos principais objetivos de usar controlo magnético em tokamaks é controlar a posição e a forma do plasma, seja para levar a posição do plasma a uma dada referência ou para rejeitar as perturbações que possam  ocorrer e manter a forma do plasma num determinado equilíbrio pré-definido. Isto é possível variando as correntes e tensões das bobines de campo poloidal  (PF coils em inglês), através de processamento dos sinais gerados pelas sondas magnéticas ou bobinas de Mirnov com o fim de reconstruir a posição do centroide de corrente do plasma ou a última superfície fechada de campo magnético (LCFS em inglês).\smallskip

Esta tese apresenta  uma descrição geral dos sistemas de controlo e os principais conceitos da engenharia de controlo usados nos tokamaks assim como as avaliações e melhorias realizadas para dois tokamaks: o JT60-SA (Japão) e o ISTTOK (Portugal). Estes dois dispositivos dependem do controlo ativo das bobines de campo poloidal para controlar a forma e posição do plasma. O JT60-SA é um tokamak supercondutor que ainda se encontra em construção e será o maior tokamak existente no mundo que iniciará operações no fim de 2020. O ISTTOK é um tokamak com uma proporção de aspecto grande que tem estado em operação por mais de 30 anos e é característico pela sua operação em modo de corrente alternada (AC) e a sua flexibilidade em geral.\smallskip

O capítulo introdutório da tese permite ao leitor compreender porque é precisa uma forma toroidal para confinar magneticamente o plasma, o papel dos campos magnéticos na interacção com o plasma, a forma em que é gerada a corrente do plasma, as razões pelas quais um tokamak precisa de ter bobines de campo poloidal e por que é que estas precisam de ter controlo ativo com o fim de estabilizar o plasma. Explicam-se ainda as forças experimentadas por uma partícula carregada na presença dum campo toroidal até chegar às expressões do balanço de forças toroidais. Finalmente este capítulo expõe  duma maneira simples como funciona o controlo em tokamaks e como é que se foi desenvolvendo ao longo do tempo e apresenta ao leitor como é que está organizada a tese capítulo por capítulo. \smallskip

No capítulo 2  os sistemas de controlo para plasmas são vastamente descritos, dando destaque ao framework MARTe, o qual será amplamente usado e referido nos seguintes capítulos,  assim como  os códigos de equilíbrio usados em alguns tokamaks com o fim de reconstruir parâmetros do plasma usados para atingir o controlo da posição, forma e corrente do plasma. A parte final deste capítulo está centrada  nos conceitos básicos dos sistemas lineares e invariantes no tempo assim como o desenho de sistemas de controlo, os quais são amplamente usados nos capítulos a seguir, como por exemplo os sistemas em espaço de estados, controladores PID e filtros de Kalman. \smallskip

No início de capítulo 3 o  JT60-SA é descrito assim como o cenário sobre o qual se baseiam as simulações realizadas , tal como os seus valores de equilíbrio. De  seguida descrevem-se as ferramentas de modelização do plasma CREATE e como estas estabelecem um sistema em espaço de estados para descrever o plasma ao redor de certo equilíbrio. Posteriormente aborda-se o desenho dos controladores da forma e corrente do plasma com dois enfoques diferentes: usando descritores de posição chamados gaps (do inglês para espaçamento) ou superfícies de fluxo magnético. Nas simulações deste capítulo usam-se dois controladores diferentes: o controlador desenvolvido pela equipa do QST (Japan National Institutes for Quantum and Radiological Science and Technology) e o eXtreme Shape Controller (XSC) aplicado ao JT-60SA no âmbito do trabalho desta tese, onde o JT60-SA é simulado através do modelo CREATE que reconstrói  o equilíbrio do cenário. Neste capítulo é também apresentada a validação realizada ao método utilizado pelo QST para reconstruir a última superfície fechada de campo magnético, utilizando um código chamado CCS pelo seu nome em inglês Cauchy Condition Surface. A comparação entre estes dois controladores e dos fluxos magnéticos obtidos usando o modelo CREATE e o código CCS para reconstruir a forma do plasma é realizada na presença de diferentes perturbações tais como os ELMs e com uma selecção de número de pontos a controlar distinta para cada caso. Cada um dos controladores parece ter a suas próprias vantagens e desvantagens. Com o fim de testar a flexibilidade do controlador XSC uma simulação adicional foi realizada, que consiste em mudar a  referência da forma do plasma na parte superior do tokamak  durante a simulação com um tempo de transferência duma forma do plasma  à outra de 1.5 segundos. \smallskip

O capítulo 4 começa com uma exaustiva descrição do ISTTOK: breve história do tokamak, os diagnósticos, os atuadores e a geometria dele. Posteriormente aprofunda-se a descrição da corrente do plasma no ISTTOK e as transições que ela tem entre corrente positiva e negativa, sem perder a ionização do plasma quando a corrente é praticamente zero, atingindo assim uma descarga muito mais comprida com corrente de plasma AC. A seguir  são descritas as novas implementações de hardware feitas pela equipa do Instituto de Plasmas e Fusão Nuclear (IPFN)  as quais consistem na adição de integradores numéricos, os quais permitem integrar os sinais provenientes das sondas magnéticas em tempo-real antes destes serem digitalizados e processados no framework MARTe. Esta implementação é fundamental para reconstruir em tempo real a posição do centroide da corrente de plasma.  Depois de condicionar o sinal com o fim de remover offsets foi implementado através de uma modelização experimental um algoritmo para subtrair das medidas integradas das sondas magnéticas a contribuição da corrente de plasma. Desta forma é possível dividir dos sinais medidos a contribuição magnética do plasma daquela que é gerada pelas bobinas de campo poloidal. Posteriormente é usado um modelo de multi-filamentos para modelar o plasma a partir dos sinais limpos das sondas com o fim de obter a posição do centroide da corrente do plasma em tempo real. Os modelos baseados em multi-filamentos para descrever o plasma têm sido amplamente usados e estudados ao longo do tempo.  Finalmente é feita  uma comparação da forma como com a nova implementação da reconstrução da posição do centroide do plasma é possível ter transições bem-sucedidas entre plasma negativo e positivo. \smallskip

Finalmente no capítulo 5 os resultados obtidos das novas implementações de controlo em tempo real são mostradas e analisadas. Começa-se por descrever rapidamente como é que a implementação de algoritmos de controlo interactua com o framework MARTe assim como explicar como o ciclo de controlo é fechado desde que os sinais das sondas magnéticas são adquiridos até que os sinais de controlo em corrente são injetados nas fontes de tensão das bobinas de campo poloidal. Dadas as características reais geométricas e de construção do ISTTOK existem atualmente poucas possibilidades de ter um modelo linear e teórico a relacionar a posição do centroide com as correntes nas bobinas de campo poloidal. Este facto além de ser um impedimento para desenvolver um controlador Multiple-Input Multiple-Output (MIMO) foi um incentivo para procurar outras alternativas. No ISTTOK optou-se por fazer uso das ferramentas computacionais atuais e usar um modelo MIMO reconstruído a partir de dados experimentais. A partir deste modelo foi possível programar no framework MARTe um controlador ótimo em conjunto com uma série de controladores PID, os quais foram sintonizados empiricamente. No fim deste capítulo  uma série de descargas de plasma no ISTTOK são comparadas. Nelas compara-se o desempenho dos controladores  PIDs e do controlo ótimo MIMO em termos da posição do centroide do plasma e da quantidade de corrente requerida pelas fontes de tensão das bobinas de campo poloidal. \smallskip

Juntamente com as melhorias de controlo atingidas para os dois tokamaks, nesta tese, um dos objetivos atingidos foi a aplicação do conhecimento obtido nas simulações feitas para o JT60-SA num ambiente real, podendo ser verificados experimentalmente no ISTTOK.



\vfill

\end{document}