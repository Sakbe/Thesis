\chapter{Plasma Control Systems}

\section{Overview of control systems}
The control of  plasma position, shape and current among other parameters is one of the crucial engineering problems for present and future magnetic confinement devices. The Plasma Control Systems (PCS) lead with the overall control of  fusion devices being responsible also for the  plasma configuration and scenarios algorithms \cite[Chapter~8]{PCS_2018}. Currently different PCS's are use in the tokamaks around the world. In this chapter the "DIII-D-like" PCS, the Syst\'eme de Contr\^ole Distribu\'e (SCD) and the Multi-threaded Application Real-Time executor (MARTe) will be approach, this last one being of special interest due to its extensive utilization in this work.

\subsection{DIII-D Plasma Control System}  

The DIII-D-like PCS is use in various fusion research facilities such as EAST(China), K-STAR (South Korea) and MAST (UK). Early documentation regarding the PCS in DIII-D\footnote{DIII-D is a D-shape tokamak operated by General Atomics in San Diego, California. } reefers to digitalization of analog signals transmitted to a high speed processor executing a shape control algorithm and then writing the result to a digital to analog converter for driving the controlled systems . The real-time computer used allowed to performed operations with vectors and matrices required for the plasma shape control algorithm \cite{DIIDcontrol}. Figure ~\ref{DIII1991} shows the block diagram of the DIII-D PCS 30 years ago.
\smallskip

\begin{figure}[htbp]
	\centering
	\includegraphics[width=0.65\textwidth]{Chp2/DIIDPCS_old.PNG}
	\caption{\label{DIII1991} DIII-D digital PCS in 1991 ~\cite{DIIDcontrol}.  }
\end{figure}

In recent years the DIII-D PCS had extensive software and hardware upgrades. The PCS actual software consists of an infrastructure library core which provides all the routines that are necessary for implementing a basic and generic control system. The current  PCS hardware configuration uses a collection of  Intel Linux based multi-processor computers running in parallel to perform the real-time analysis and feedback control ~\cite{DIIID2013}. New digitizers have been added to the real-time network to increase the number of signals acquired an to control hardware on real-time, several real-time control algorithms were added and real-time data was added to external entities such as web server.~\cite{DIIIDnew}. In the current version of the PCS, a Myricom\footnote{Myricom networks also called Myrnet are high speed networking systems used to interconnect machines to form computer clusters. } network has been replaced with a 40 Gb/sec InfiniBand\footnote{Is a network architecture from Mellanox designed to support I/O connectivity  and  reliability, availability, and serviceability Internet requirements ~\cite{MellanoxTechnologies2003}.  } network based on the Mellanox Connect-X 3\footnote{The Connect-X from the Mellanox company are Ethernet network interface cards with PCI Express.} hardware set. Figure ~\ref{DIIInew} shows the currently overall networking diagram of DIII-D PCS .


\begin{figure}[htbp]
	\centering
	\includegraphics[width=0.65\textwidth]{Chp2/DIIIDPCSnew.PNG}
	\caption{\label{DIIInew} Actual DIII-D PCS real-time systems ~\cite{DIIIDnew}.  }
\end{figure}


\subsection{Syst\'eme de Contr\^ole Distribu\'e}

The TCV\footnote{The Tokamak \'a configuration variable (TCV) is  a medium size tokamak localized in Laussane,Switzerland. It is characterized by a highly elongated, rectangular vacuum vessel.} distributed control system uses a modular network of real time PC nodes liken by a real time network to provide feedback control over all of the actuator systems. Each node consists of a Linux PC either embedded on a Compact-PCI module or as a desktop computer with Intel CPU. A fiber optic ring network links the reflective memory (RFM) network cards in each node  \cite{TCVcntrl}.  The design of the diagnostic signal processing and control algorithms is performed in Matlab-Simulink software.  During the real-time execution  C/C++  code is generated from the Simulink and compiled  into a Linux shared library and distributed to target nodes  providing the input/output interface to the control algorithm code  ~\cite{TCVcntrl1}. Figure ~\ref{TCVcontrol} depicts the TCV SCD layout with the connectivity to diagnostics and actuators.


\begin{figure}[htbp]
	\centering
	\includegraphics[width=0.65\textwidth]{Chp2/TCVcntrl1.png}
	\caption{\label{TCVcontrol} TCV SCD. Real-time network nodes connection. The nodes configurations 	are shown together with the typical diagnostic and actuator systems to which they are connected  ~\cite{TCVcntrl1}.  }
\end{figure}

\section{MARTe framework}

MARTe was developed in order to standardize general real-time control systems for the execution of control algorithms. MARTe framework is based on a multiplatform $C^{++}$ library. \cite{Neto2011} 

\subsection{MARTe architecture }
\subsection{Hardware containers}
\subsection{MARTe 2.0}
\section{Equilibrium and control algorithms} 
\subsection{PID control}
\subsection{Multiple-Input Multiple-Output control}