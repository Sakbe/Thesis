\chapter{Plasma Control Systems}

\section{Overview of control systems}
The control of  plasma position, shape and current among other parameters is one of the crucial engineering problems for present and future magnetic confinement devices. The Plasma Control Systems (PCS) lead with the overall control of the fusion devices  being responsible also for the  plasma configuration and scenarios algorithms \cite[Chapter~8]{PCS_2018}. 
\hfil

Currently different PCS's are use in the tokamaks around the world. In this chapter will be approach the "DIII-D-like" PCS and the Multi-threaded Application Real-Time executor (MARTe).

\subsection{DIII-D Plasma Control System}  

\cite{DIIDcontrol}

\section{MARTe framework}

MARTe was developed in order to standardize general real-time control systems for the execution of control algorithms. MARTe framework is based on a multiplatform $C^{++}$ library. \cite{Neto2011} 

\subsection{MARTe architecture }
\subsection{Hardware containers}
\subsection{MARTe 2.0}
\section{Equilibrium and control algorithms} 
\subsection{PID control}
\subsection{Multiple-Input Multiple-Output control}