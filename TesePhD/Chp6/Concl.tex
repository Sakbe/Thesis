\chapter{Conclusions}


This thesis consists in a deep study of the control algorithms and systems used in two tokamaks: ISTTOK and JT60-SA. Despite the big differences between them is not difficult to find the same linking line of study for both cases.\smallskip

Novelty techniques must be consider for future control implementations in the field of controlled nuclear fusion  such as machine learning and neuronal networks, this ones being already  used by researchers  attempting to detect disruptions before they happen so that they can be stopped in order to avoid catastrophic wall damage~\cite[Chapter~6]{Paluszek2020}. This conclusion chapter is split in two section: one dedicated to JT60-SA and another to ISTTOK, addressing the conclusions  this work brought for each device and possible future work.

\section{JT60-SA}

As mentioned before the big importance of JT60-SA  lies primarily in the fact that it will be the biggest operating tokamak. From the results showed in Chapter 2 both controllers, the XSC and the QST controller are suitable  . \smallskip 

The decision of which control approach   will be used in JT60-SA which will start operating in 2020.

\section{ISTTOK}

Experimental results presented in Chapter 3 and 4 show that the MARTe framework along with the new numerical integrators provide the adequate tools for developing the ISTTOK tokamak real-time control. The implementation of the centroid position reconstruction based on a multi-filament model demonstrated to be  \smallskip

It is worth to mentioned how retrieving the plasma magnetic poloidal field and obtaining a state-space model linking the plasma position and the Pf coils currents required the use of computational tools.\smallskip

Due to its characteristics widely discussed in this work,ISTTOK is a tokamak that might bring  more challenges when it comes to implement the tokamak physics than in other devices but it is also a very flexible machine which allows to test new methods and approaches without risking the tokamak itself.\smallskip

Machine learning is a rapidly developing field that is transforming our ability to describe complex systems from experimental data, rather than theoretical principles for modeling them. As machine learning encompasses a broad range of high-dimensional, possibly nonlinear, optimization techniques, it is natural to apply machine learning to the control of complex systems like a tokamak ~\cite[Chapter~10]{DataDriven2019}. Future work in ISTTOK based on this principles such as the iterative learning control which is an effective control tool for improving the transient response and tracking performance of uncertain dynamic systems that operate repetitively~\cite{Ahn2007}, must be consider for future upgrade of the plasma position control.\smallskip

