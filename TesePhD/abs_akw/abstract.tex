\chapter*{Abstract}
Magnetic control for fusion plasmas is one of the main engineering tasks to be solved in magnetically  confined devices like tokamaks. Magnetic control is the tool that allows to control the plasma position and shape in a tokamak, whether  for steering the plasma position to a given set point or rejecting disturbances which may occur and maintain the plasma shape in a certain equilibrium. Such goals are achieved by varying the currents that are driven on the Poloidal Field (PF)\footnote{Sometimes the name PF coils is used to refer to  both the equilibrium field coils and the ohmic heating coils for generating plasma current.} coils while monitoring several diagnostics that allow the reconstruction of the plasma current, position and its last closed flux surface (LCFS) in a real-time feeddback acquisition and control system.
\smallskip

This thesis presents a comprehensive overview of control systems and some of the main control engineering concepts used in tokamaks along with  the assessments and upgrades performed for two tokamaks: JT-60SA (Japan) and ISTTOK (Portugal). These two devices rely in the active control of the PF coils to control the plasma shape and position. JT-60SA is an under construction superconductive tokamak that will become the largest one built so far and will start operating in late 2020. ISTTOK is a large aspect ratio tokamak operating for 30 years and it is characterized by its AC operation mode and flexibility. \smallskip

 Along with presenting the achieved control assessments for both devices in this thesis, one of the main objectives  is also that the simulation work done for JT-60SA can be confirmed in an experimental sense in ISTTOK. \smallskip

 
The JT-60SA work done in this thesis consists in a series of simulations testing two different shape controllers and  approaches for obtaining the LCFS  in the presence of several disturbances and a change of the reference plasma shape, along with the comparison of results obtained from both controllers and the flux data from the LCFS. The implementation of these controllers is possible by using a given theoretical equilibrium in the form of a linear state-space model for the simulation of the overall magnetic  tokamak behavior. \smallskip

 The work developed in ISTTOK consisted in the application of several physics concepts and computational tools in order to have a novel optimal controller and a plasma centroid position reconstruction implemented on real-time. Recently upgraded hardware numerically integrates  the magnetic probes signals which are acquired on real-time, being this fact a key point for the development of this part of the thesis .\smallskip
 

 

Each tokamak is addressed for different aims and under a different scope in this work. The JT-60SA work benchmarks the CREATE magnetic modeling tools against the official QST tools, which opens up the possibility of considering the CREATE tools as a possible backup to support the optimization of the controller for JT-60SA operation. ISTTOK work demonstrates that the used MARTe framework and ATCA hardware architecture, along with the new numerically integration hardware implementation, provide a set of adequate tools for developing the ISTTOK tokamak real-time control and plasma centroid position reconstruction
 


\textbf{Keywords:Real-time control, plasma current, plasma current centroid position, state-space model, optimal control, shape control, magnetic probe, PF coil, last closed flux surface(LCFS), numerical integration.} 

\pagebreak
\begin{otherlanguage}{portuguese}
\chapter*{Resumo}


O controlo magnético para plasma de fusão é uma das principais tarefas a ser desenvolvida em dispositivos de confinamento magnético como os tokamaks. O controlo magnético é uma ferramenta que permite controlar a posição e forma do plasma nos tokamaks, seja para levar a posição do plasma a uma referência dada  ou para rejeitar perturbações que possam ocorrer e manter a forma do plasma num certo equilíbrio.  Estes objectivos conseguem se variando as correntes das bobines de campo poloidal (PF coils em inglês) \footnote{Às vezes o nome de PF coils se refere às bobines do campo de equilíbrio assim como as bobines para aquecimento ohmico responsáveis de gerar a corrente do plasma.} assim como também com a  monitorização de vários diagnósticos os quais permitem a reconstrução da corrente do plasma, da posição e da última superfície fechada de fluxo (LFCS em inglês) num sistema em tempo-real de aquisição de dados e de controlo em laço fechado.\smallskip

Esta tese apresenta uma descrição completa dos sistemas de controlo e alguns dos principais conceitos da engenharia de controlo usados nos tokamaks assim como as melhorias e atualizações realizadas para dois tokamaks: o JT-60SA (Japao) e o ISTTOK (Portugal). Estes dois dispositivos dependem do controlo ativo das bobines de campo poloidal para controlar a forma e posição do plasma. O JT60-SA é um tokamak supercondutor que ainda se encontra em construção e será o maior tokamak existente no mundo que iniciará operações no fim de 2020. O ISTTOK é um tokamak com uma razão de aspecto grande que tem estado em operação por mais de 30 anos e é característico pela sua operação em modo de corrente alternada (AC) e a sua flexibilidade em geral.\smallskip

Em conjunto com a apresentação dos resultados de controlo atingidos para os dois dispositivos nesta tese, um dos principais objectivos é também que o trabalho de simulação feito para o JT60-SA possa ser confirmado num sentido experimental no ISTTOK.\smallskip

O trabalho feito para o JT60-SA nesta tese consiste numa série de simulações experimentado com dois controladores diferentes para a forma do plasma e métodos para obter a última superfície fechada de fluxo na presença de distintas perturbações e de uma mudança na referência da forma do plasma assim como a comparativa dos resultados obtidos destes dois controladores e  das medidas de fluxo da última superfície fechada de fluxo. A implementação deste controladores é  possível por meio dum equilíbrio teórico dado na forma dum modelo linear em espaço de estados do comportamento magnético do tokamak.
\smallskip

O trabalho desenvolvido no ISTTOK consistiu na aplicação de diferentes conceitos físicos e ferramentas computacionais  para obter um novo controlador ótimo e uma reconstrução do centróide da corrente do plasma em tempo real. O recentemente atualizado hardware faz integração numérica dos sinais das sondas magnéticas as quais são adquiridas em tempo-real, sendo este fato uma peça chave no desenvolvimento desta parte da tese.\smallskip

Cada um dos tokamaks é abordado para diferentes objectivos e sob uma luz diferente neste trabalho. O trabalho feito para o JT60-SA compara as ferramentas magnéticas de modelização CREATE  com as ferramentas oficiais QST, o que representa uma possibilidade de considerar as ferramentas CREATE como uma possível reserva para apoiar a otimização do controlador na operação do JT60-SA. O trabalho feito no ISTTOK demonstra que o uso de framework MARTe e da arquitetura de hardware ATCA, em conjunto com a implementação do novo hardware para integração numérica, proporciona um conjunto de ferramentas adequadas para desenvolver controladores e reconstruir a posição do centróide da corrente do plasma em tempo-real.
\smallskip

\vfill
\textbf{ Palavras-chave: Controlo em tempo real, corrente do plasma, posição do centróide da corrente do plasma, modelo em espaço de estados, controlo óptimo, controlo da forma do plasma, sonda magnética, bobina de campo poloidal(PF coil), última superfície fechada de fluxo(LCFS), integração numérica.    } 

\end{otherlanguage}
\pagebreak
\begin{otherlanguage}{italian}
\chapter*{Sommario}

Abstract em italiano


\vfill

\textbf{Parole chiave:} 
\end{otherlanguage}



